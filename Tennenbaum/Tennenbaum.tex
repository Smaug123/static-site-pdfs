\documentclass[11pt]{amsart}
\usepackage{geometry}
\geometry{a4paper}
\usepackage{graphicx}
\usepackage{amssymb}
\usepackage{mdframed}
\usepackage{hyperref}
\usepackage{lmodern}

% Reproducible builds
\pdfinfoomitdate=1
\pdftrailerid{}
\pdfsuppressptexinfo=-1

\newmdtheoremenv{thm}{Theorem}[section]

\theoremstyle{remark}
\newtheorem*{note}{Notation}
\newtheorem*{rmk}{Remark}
\newtheorem*{example}{Example}

\title{Tennenbaum's Theorem}
\author{Patrick Stevens}
\date{27th April 2016}

\begin{document}

\maketitle 
\tiny \begin{center} \url{https://www.patrickstevens.co.uk/misc/Tennenbaum/Tennenbaum.pdf} \end{center}
\normalsize

\section{Introduction}

\begin{thm}[Tennenbaum's Theorem]
Let $\mathfrak{M}$ be a countable non-standard model of Peano arithmetic, whose carrier set is $\mathbb{N}$.
Then it is not the case that $+$ and $\times$ have decidable graphs in the model.
\end{thm}

\

\begin{note}
We will use the notation $\{ e\}$ to represent the $e$th Turing machine.
$e$ is considered only to be a standard integer here.
For example, we might view the G\"odel numbering scheme as being ``convert from ASCII and then interpret as a Python program''.
\end{note}

\begin{rmk}
How might our standard Turing machine refer to a nonstandard integer?
The ground set of our nonstandard model is $\mathbb{N}$: every nonstandard integer has a standard one which represents it in $\mathbb{N}$.
Perhaps $4 \in \mathbb{N}$ is the object that the nonstandard model $\mathfrak{M}$ thinks is the number $7$, for instance.
So the way a Turing machine would refer to the number $7$-in-the-model is to use $4$ in its source code.
\end{rmk}

What does it mean for $+$ to have a decidable graph?
Simply that there is some (standard) natural $n$ such that,
when we unpack $n$ into instructions for running a Turing machine,
we obtain a machine that takes three naturals (that is, standard naturals) $a, b, c$ and outputs $1$ iff, when we take the referents $a', b', c'$ of $a, b, c$ in the model $\mathfrak{M}$, it is true that $a' +_{\mathfrak{M}} b' = c'$.

\begin{example}
A strictly standard-length program may halt in nonstandard time, when interpreted in a nonstandard model.
Indeed, fix some nonstandard ``infinite'' $n$ (i.e. $n$ is not a standard natural).
Then the following program halts after $n$ steps.
\begin{verbatim} 
ans = 0;
for i = 1 to n:
  ans := ans + 1;
end
HALT with output ans;
\end{verbatim}
\end{example}

\section{Overview of the proof}

The proof \emph{est omnis divisa in partes tres}.

\begin{enumerate}
\item In any model, there is some pair of semidecidable but recursively inseparable sets.
\item We can use these to create an undecidable set of true standard naturals which can, in some sense, be coded up into a (nonstandard) natural in our model.
\item If $+$ and $\times$ were decidable, then the coding process would produce an object which would let us decide the undecidable set; contradiction.
\end{enumerate}

\section{Existence of recursively inseparable sets}
This is fairly easy. Take $A = \{ e : \{ e \}(e) \downarrow = 0 \}$ and $B = \{ e: \{e \}(e) \downarrow > 0 \}$, where $\downarrow=$ means ``halts and is equal to'', and $\downarrow >$ means ``halts and is greater than''.
Recall that $e$ must be standard.

Now, suppose there were a (standard) integer $n$ such that $\{ n \}$ were the indicator function on set $X$, where $X \cap B = \emptyset$ and $A \subseteq X$.
Then what is $\{n\}(n)$?
If it were $0$, then $n$ is not in $X$, so $n$ is not in $A$ and so $\{n\}(n)$ doesn't halt at $0$.
That's a contradiction.
If it were $1$, then $n$ is in $X$ and hence is not in $B$, so $\{n\}(n)$ doesn't halt at something bigger than $0$; again a contradiction.

So we have produced a pair of sets which are both semidecidable but are recursively inseparable, in the sense that no standard integer $n$ has $\{n\}$ deciding a superset $X$ of $A$ where $X \cap B = \emptyset$.
(This is independent of the model of PA we were considering; it's purely happening over the ground set.)

\section{Coding sets of naturals as naturals}
We can take any set of (possibly nonstandard) naturals and code it as a (possibly nonstandard) natural, as follows.
Given $\{ n_i : i \in I \}$, code it as $\sum_{i \in I} 2^{n_i}$.
If $+$ and $\times$ are decidable, then this is a decidable coding scheme.
(The preceding line is going to be where our contradiction arises, right at the end of the proof!)

Notice that if $I$ is ``standard-infinite'' (that is, it contains nonstandardly-many elements) then the resulting code is nonstandard.
Additionally if any $n_i$ is strictly-nonstandard.

\section{Undecidable set in \texorpdfstring{$\mathfrak{M}$}{M}}
Take our pair of recursively inseparable semidecidable sets: $\mathfrak{A}$ and $\mathfrak{B}$.
(We constructed them explicitly earlier, but now we don't care what they are.)
Recalling a theorem that being semidecidable is equivalent to being a projection of a decidable set, 
write $A$ for a decidable set such that $(\exists y)[(n, y) \in A]$ if and only if $n \in \mathfrak{A}$,
and similarly for $B$.
(The quantifiers range over $\mathbb{N}$, because $A$ and $B$ consist only of standard naturals, being subsets of the ground set.)

By their recursive-inseparability, they are in particular disjoint, so we have $$(\forall n)[(\exists x)(\langle n, x \rangle \in A) \to \neg (\exists y)(\langle n, y \rangle \in B)]$$
where the quantifiers all range over $\mathbb{N}$.
Equivalently, $$(\forall n)(\forall x)(\forall y)(\neg \langle n,x \rangle \in A \vee \neg \langle n,y \rangle \in B)$$
If we bound the quantifiers by any standard $m = SS\dots S(0)$ (which we explicitly write out, so it's absolute between all models of PA), we obtain an expression which our nonstandard model believes, because the expression is absolute for PA:
$$(\forall n < m)(\forall x < m)(\forall y < m)(\neg \langle n,x \rangle \in A \vee \neg \langle n,y \rangle \in B)$$

This is true for every standard $m$, and so it must be true for some nonstandard $m$ by overspill, since $\mathfrak{M}$ doesn't know how to distinguish between standard and nonstandard elements.
If the property were only ever true for standard $m$, then $\mathfrak{M}$ could identify nonstandard $m$ by checking whether that property held for $m$.

Let $e$ be strictly nonstandard such that 
\begin{equation} \label{eqn:prop}
\mathfrak{M} \vDash (\forall n < e)(\forall x < e)(\forall y < e)(\langle n,x \rangle \not \in A \vee \langle n,y \rangle \not \in B)
\end{equation}
where we note that this time $e$ is not written out explicitly as $SS\dots S(0)$ because it's too big to do that with.

Finally, we define our undecidable set $X \subseteq \mathbb{N}$ of \emph{standard} naturals to be those standard naturals $x$ such that $$\mathfrak{M} \vDash (\exists y < e) (\langle x, y \rangle \in A)$$
This is undecidable in the standard sense: there are no standard $m$ such that $\{m \}$ is the indicator function of $X$.
Indeed, I claim that $X$ separates $\mathfrak{A}$ and $\mathfrak{B}$.
(Recall that all members of $X$, $\mathfrak{A}$ and $\mathfrak{B}$ are standard.)

\begin{itemize}
\item If $a \in \mathfrak{A}$ then there is some standard natural $n$ such that $\langle a, n \rangle \in A$;
and $n$ is certainly less than the nonstandard $e$.
Hence $a \in X$.
\item If $b \in \mathfrak{B}$, then there is standard $n$ such that $\langle b, n \rangle \in B$.
Then $n < e$, so by (\ref{eqn:prop}) we have $\langle b, x \rangle \not \in A$ for all $x < e$.
That is, $b \not \in X$.
\end{itemize}

\section{Coding up \texorpdfstring{$X$}{X}}
Now if we code up $X$, which is undecidable, using our coding scheme $$\{ n_i : i \in I \} \mapsto \sum_{i \in I} 2^{n_i}$$
we obtain some nonstandard natural; say $p = \sum_{x \in X} 2^x$.
Supposing the $+$ and $\times$ relations to be decidable, this coding is decidable.
Remember that $X$ is a set of standard naturals which is undecidable: no standard Turing machine decides $X$.

But here is a procedure to determine whether a standard element $i \in \mathbb{N}$ is in $X$ or not:

\begin{enumerate}
\item Take the $i$th bit of $p$. (This is decidable because $+$ and $\times$ are.)
\item Return ``not in $X$'' if the $i$th bit is $0$.
\item Otherwise return ``is in $X$''.
\end{enumerate}

This contradicts the undecidability of $X$.

\section{Acknowledgements}
The structure of the proof is from Dr Thomas Forster's lecture notes on Computability and Logic from Part III of the Cambridge Maths Tripos, lectured in 2016.

\end{document}
