\documentclass[11pt]{amsart}
\usepackage{geometry}
\geometry{a4paper}
\usepackage{graphicx}
\usepackage{amssymb}
\usepackage{epstopdf}
\usepackage{mdframed}
\usepackage{hyperref}
\usepackage{tikz-cd}
\usepackage{lmodern}

% Reproducible builds
\pdfinfoomitdate=1
\pdftrailerid{}
\pdfsuppressptexinfo=-1

\DeclareGraphicsRule{.tif}{png}{.png}{`convert #1 `dirname #1`/`basename #1 .tif`.png}

\newmdtheoremenv{defn}{Definition}
\newmdtheoremenv{thm}{Theorem}
\newmdtheoremenv{motiv}{Motivation}

\title{Representable functors and adjoint functors}
\author{Patrick Stevens}

\begin{document}

\maketitle

\tiny \begin{center} \url{https://www.patrickstevens.co.uk/misc/RepresentableFunctors/RepresentableFunctors.pdf} \end{center}

\normalsize

\emph{You should draw diagrams yourself throughout this document. It will be unreadable as mere symbols.}

\begin{defn}A \emph{representable functor} is a functor $F: \mathcal{C} \to \mathbf{Sets}$ which is naturally isomorphic to $\text{Hom}_{\mathcal{C}}(A, \cdot)$ for some $A \in \mathcal{C}$.\end{defn}

\

It is an easy fact that representable functors preserve limits.
Indeed, it is enough to show that they preserve products and equalisers, and both of these facts simply fall out of the relevant diagrams.

What other functors preserve limits?
Recall that morally, a functor has a left adjoint iff it preserves all limits.
(There are set-theoretic issues here, so the statement isn't actually true as stated; these are dealt with by the General Adjoint Functor Theorem.)

So we might hope that representable functors into $\mathbf{Sets}$ are precisely those which are right adjoints.

\section{Right adjoints are representable}

Suppose $G: \mathcal{C} \to \mathbf{Sets}$ is a right adjoint to $F: \mathbf{Sets} \to \mathcal{C}$.
We want to find a representing object, and it's going to have to use $F$, so let's see what would happen if $G$ were naturally isomorphic to $\text{Hom}_{\mathcal{C}}(FA, \cdot)$.

By the definition of an adjoint, for every $A \in \mathcal{C}$, have $$\text{Hom}_{\mathcal{C}}(FA, X) \cong \text{Hom}_{\mathbf{Sets}}(A, GX)$$

But wait! We already have that $\text{Hom}_{\mathbf{Sets}}(\{ 1 \}, GX) \cong GX$.
So actually $F1$ represents $G$.

\section{Representable functors have left adjoints}

This one is less immediate, so we'll motivate it.
The most trivial example of a representable functor is $1_{\mathbf{Sets}}$, so we'll go for the next most trivial: $$\text{Hom}_{\mathbf{Sets}}(\{0, 1\}, \cdot) : \mathbf{Sets} \to \mathbf{Sets}$$

What would it look like if this had an adjoint?
We'll use the most concrete definition of an adjoint: as an initial object in an appropriate comma category.

\[
\begin{tikzcd}
FA
    \arrow[r, dashrightarrow, "h"]
& B
\\
\mathrm{Hom}(\{0,1\}, FA)
    \arrow[r, "h \circ \cdot"]
& \mathrm{Hom}(\{0,1\}, B)
\\
A
    \arrow[u, "\eta_A"]
    \arrow[ur, "f"']
&
\end{tikzcd}
\]

What is $f: A \to \text{Hom}(\{0, 1\}, B)$?
It is nothing more than an $A$-indexed collection of maps $\{0, 1\} \to B$.

We want to make this unique $h: FA \to B$, given a collection of maps $\{0,1\} \to B$.

That suggests we want to glue the maps together somehow, and $FA$ will be a ``glued object''.

So we just take $FA$ to be the coproduct of $\{0,1\}$-many copies of $A$.

That is, let $FA = \sqcup_{\{0, 1\}} A = \{a_0: a \in A\} \cup \{a_1: a \in A\}$, and define $\eta_A: A \to \text{Hom}(\{0,1\}, FA)$ by $a \mapsto (n \mapsto a_n)$.

This construction generalises in exactly the obvious way to any representable functor $\mathbf{Sets} \to \mathbf{Sets}$, and then to $\mathcal{C} \to \mathbf{Sets}$ where $\mathbf{C}$ has coproducts.

\end{document}